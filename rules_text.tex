\marginnote{\section*{Gameplay}
It is important that everyone understands all of the following instructions before you begin. To play Mugshots you should:
\begin{enumerate}[leftmargin=*]
\item Partition the players into groups with two or three players in each group.
\item Start the timer. Try to memorize the portraits that appear on your player sheet.
\item When time expires, turn your player sheet over so that you cannot see the portraits displayed thereon.
\item Describe the portraits that appear on your sheet to the other players in your group. Try to identify a portrait that appears on more than one of your player sheets.
\item Finally, each player should simultaneously choose one of the portraits on the Mugshots Gallery. If everyone in your group chose a portrait that appears on more than one of your player sheets, you win! 
\end{enumerate}
}[-15.5875cm]
\reversemarginpar\marginnote{\raggedright\section*{Overview}
Mugshots is a cooperative find-the-match game for two or more players which can be played in less than ten minutes.

Each player will need a player sheet with a unique \emph{sequence number}. You will need one copy of the Mugshots Gallery and a two-minute sand timer to share.

\textbf{Note:} You should not show your player sheet to any other player.

There are seventy-three different player sheets. Nine portraits are displayed on each player sheet. For any two player sheets, one portrait appears on both sheets.

Seventy-seven portraits are displayed on the Mugshots Gallery. Every portrait that appears on a player sheet also appears on the Mugshots Gallery.

During the game you will try to memorize the portraits that appear on your player sheet. Then, you will work with other players to identify which portraits you have in common.
}[-15.5875cm]
